\documentclass[dvips,letterpaper]{article}
\usepackage[dvips]{graphicx}

% packages
\usepackage{hyperref}
\usepackage{booktabs}

% results dir
\newcommand{\resultsdir}{../../results/2014-05-23/}

% doc info
\author{Kristofer D. Kusano}
\title{Flight Delays at ROA and IAD Airports}
\date{May, 25 2014}

\begin{document}
% abstract
\maketitle

\begin{abstract}
  Flying out of the Roanoke Region Airport (code ROA) has its challenges.
  It seems like flights going to and from small airports like ROA are frequently delayed.
  Bigger airports, like Washington Dulles International Airport, might not suffer from the same types of delays.
  This white paper compares delay times between ROA and IAD using a record of all flights originating from or traveling to the two airports.
\end{abstract}

\section{Methods}
\label{sec:Methods}
Flight records were retrieved from the Research and Innovation Transportation Agency (RITA), Bureau of Transportation for Statistics (BTS).
The BTS provides data on departure and arrival times for flights originating and departing from commercial airports in the U.S.\footnote{Data are available for download from
  \url{http://www.transtats.bts.gov/DL_SelectFields.asp?Table_ID=236)}}

In an attempt to limit confounding factors, two months were selected that were representative of a high and low delay time periods.
Flight records were downloaded for the entire months of September 2013 and February 2014.
September was selected as a month without holiday travel and without severe weather.
February was selected as a month with bad weather that may have had higher than usual delays.
The winter of 2013-2014 featured the ``polar vortex'' weather phenomenon that caused multiple delays.
\section{Results}
\label{sec:Results}

\subsection{Selected Flights}
\label{sub:select}
The number of flights to or from ROA in February 2014 and September 2013 is shown in Table~\ref{tab:nflights}.
\input{\resultsdir tab_nflights.tex}

The number of canceled flights was much higher in February than in September.
Table~\ref{tab:cancelled_by_month} shows the number of flights that were cancelled in each month at both the airports.
The proportion of cancelled flights was approximately 10 times higher in February than in September.
\input{\resultsdir tab_cancelled_by_month.tex}

\subsection{Cancelled Flights at ROA and IAD}
\label{sub:canelled}
Figure~\ref{fig:cancelled_by_month} shows the percentage of flights that were cancelled involving both airports.
In February, slightly more flights were cancelled out of ROA and IAD.
The reverse trend was observed in September, where almost no flights were cancelled from ROA.
\begin{figure}[!htbp]
  \centering
  \includegraphics{\resultsdir cancelled_by_month.eps}
  \caption{Cancelled Flights by Month at ROA and IAD}
  \label{fig:cancelled_by_month}
\end{figure}

\subsection{Delay Times}
\label{sub:delay_cdf}
Figure~\ref{fig:ecdf_delays} shows the cumulative distribution of delay times at each airport.
A negative delay time corresponds to a flight that left before the scheduled departure time.
Overall, only half of all flights depart before or at the scheduled departure time.
When the weather is bad (February), departure delays were similar between airports.
When weather was good (September), ROA had fewer delays.

\begin{figure}[!htbp]
  \centering
  \includegraphics{\resultsdir delay_time_cdfs.eps}
  \caption{Delay Times}
  \label{fig:ecdf_delays}
\end{figure}

\section{Conclusions}
\label{sec:Conclusions}
During good weather, ROA appears to have lower cancellation and more flights that leave on time compared to IAD.
During poor weather, however, ROA experienced more cancellations but similar delays as IAD.
Maybe our little airport is not so bad after all\ldots
\end{document}
